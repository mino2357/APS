\documentclass[10pt,a4paper,fleqn]{jsarticle}

\usepackage{amsmath,amssymb}
\usepackage{bm}
\usepackage{graphicx}
\usepackage{ascmac}
\usepackage{enumerate}
\usepackage{cases}
%
\setlength{\topmargin}{-0.3in}
\setlength{\oddsidemargin}{0pt}
\setlength{\evensidemargin}{0pt}
\setlength{\textheight}{46\baselineskip}
\setlength{\textwidth}{47zw}

\title{確率}
\date{}
\author{}

\begin{document}
\maketitle

\section{基本問題}

\begin{enumerate}
\item 袋Aには$1$から$8$までの数字が$1$つずつ書かれたカードが$8$枚入っており、袋Bには$1$から$6$までの数字が$1$つずつ書かれたカードが$6$枚入っている。いま、袋Aと袋Bから$1$枚ずつカードを取り出し、書かれた数字の積を$u$とする。このとき、次の確率を求めよ。
\begin{enumerate}[(1)]
\item $u$が奇数である確率。
\item $u$が$4$の倍数である確率。
\item $u<10$である確率。
\end{enumerate}\

\item $3$個のさいころを同時に投げるとき、次の確率を求めよ。
\begin{enumerate}[(1)]
\item 出た目がすべて$4$以下である確率。
\item 出た目の最大値が$4$である確率。
\end{enumerate}\

\item 男子$2$人、女子$1$人の合計$3$人でじゃんけんをするとき、次の確率を求めよ。
\begin{enumerate}[(1)]
\item 男子だけが勝つ確率。
\item $2$人だけが勝つ確率。
\item あいことなる確率。
\end{enumerate}\

\item $1$から$100$までの数字が書かれた$100$枚のカードから$1$枚を取り出すとき、取り出したカードに書かれた数字が$2$でも$3$でも割り切れない確率を求めよ。\\

\item Aは赤球2個と白球3個が入った袋を持っており、Bは赤球3個と白球2個が入った袋を持っている。いま、A、Bがともにそれぞれの持っている袋から同時に$2$個ずつ球を取り出すとき、次の各問いに答えよ。
\begin{enumerate}[(1)]
\item 2人が取り出した計4個の球のうち、赤球が3個である確率を求めよ。
\item 赤球を多く取り出した方を勝ちとするゲームを行ったとき、Aが勝つ確率を求めよ。\\
ただし、赤球を取り出した個数が同じ場合、または、ともに白球しか取り出さない場合は引き分けとする。
\end{enumerate}\

\item 数直線上の動点Pの最初の位置を原点とする。1個のさいころを投げて、1または2の目が出たときは$+2$、他の目が出たときは$-1$だけPを動かすものとする。次の確率を求めよ。
\begin{enumerate}[(1)]
\item さいころを3回投げて、Pが原点に戻る確率。
\item さいころを6回投げて、Pの位置が6以上となる確率。\newpage
\end{enumerate} 

\item AとBが続けて試合をする。AがBに勝つ確率は$\displaystyle \frac{1}{3}$で、引き分けはないものとする。先に3勝した方が優勝とするとき、次の確率を求めよ。
\begin{enumerate}[(1)]
\item 4試合目でAが優勝する確率。
\item Aが優勝する確率。
\end{enumerate}\

\item 100円硬貨1枚と50円硬貨3枚の計4枚の硬貨を同時に投げて、表の出た硬貨の合計金額だけ賞金をもらえるが、4枚とも裏の場合は400円を支払わなければならないゲームがある。このゲームに、参加料120円を支払って参加することは有利であるか不利であるかを答えよ。\newpage

\end{enumerate}

\section{模試の過去問}

\begin{enumerate}
\item 袋の中に、1の数が書かれた球が4個、2の数が書かれた球が3個、3の数が書かれた球が2個、4の数が書かれた球が1個の全部で10個入っている。この袋から同時に3個の球を取り出す。
\begin{enumerate}[(1)]
\item 取り出した球に書かれている3個の数がすべて2である確率を求めよ。
\item 取り出した球に書かれている3個の数の中に4が含まれている確率を求めよ。また、取り出した球に書かれている3個の数のうち、最大の数が1である確率を求めよ。
\item 取り出した球に書かれている3個の数のうち、最大の数をXとする。Xの期待値を求めよ。
\begin{flushright}
(2009年度進研模試 2年7月)
\end{flushright}
\end{enumerate}\

\item 1つのさいころを1回投げ、1、2の目が出れば1点、それ以外の目が出れば2点の得点を記録する試行を繰り返す。記録された得点の合計が6点以上となれば試行を終了し、それまでの試行回数を$X$とする。
\begin{enumerate}[(1)]
\item $X=3$となる確率を求めよ。
\item $X=4$となる確率を求めよ。
\item $X$の期待値を求めよ。
\begin{flushright}
(2009年度進研模試 2年11月)
\end{flushright}
\end{enumerate}\

\item A、B、C、D、E、Fのアルファベットが1文字ずつ書かれたカードが全部で6枚ある。6枚のカードをよく混ぜた上で、左から順に横一列に並べるとき、A、Bの文字が書かれたカードにはさまれるカードの枚数を$X$とする。ただし、A、Bの文字が書かれているカードが隣り合っている場合は$X=0$とする。
\begin{enumerate}[(1)]
\item 6枚のカードを横一列に並べる並べ方は全部で何通りあるか。
\item $X=4$となる確率を求めよ。また、$X=1$となる確率を求めよ。
\item $X$の期待値を求めよ。
\begin{flushright}
(2009年度進研模試 2年1月)
\end{flushright}
\end{enumerate}\

\item 袋Aには3枚のカード1、2、3が入っている。また、袋Bには3枚のカード2、3、4が入っている。さらに、袋Cには3枚のカード1、3、4が入っている。\\
まず、袋A、B、Cからそれぞれ1枚ずつカードを取り出す。これを1回目の操作とする。次に取り出したカードは元に戻さず袋A、B、Cからさらにそれぞれ1枚ずつカードを取り出す。これを2回目の操作とする。
\begin{enumerate}[(1)]
\item 1回目の操作で、カード3が3枚取り出される確率を求めよ。
\item 1回目の操作で、カード1が2枚取り出される確率を求めよ。
\item 1回目の操作で、カード3が3枚取り出され、かつ、2回目の操作でカード1が2枚取り出される確率を求めよ。
\item 1回目の操作で取り出された3枚のカードのうち同じカードの枚数と、2回目の操作で取りだされた3枚のカードのうち同じ数字のカードの枚数の和を$X$とする。例えば、$(3)$のときは$X=3+2=5$である。また、1回目の操作で1、2、3、1回目の操作で2、4、4が取り出されたときは$X=0+2=2$である。このとき$X=5$となる確率を求めよ。
\begin{flushright}
(2008年度第3回全統高1模試)
\end{flushright}
\end{enumerate}

\end{enumerate}






\end{document}