\documentclass[10pt,a4paper,fleqn]{jsarticle}

\usepackage{amsmath,amssymb}
\usepackage{bm}
\usepackage{graphicx}
\usepackage{ascmac}
\usepackage{enumerate}
\usepackage{cases}
%
\setlength{\topmargin}{-0.3in}
\setlength{\oddsidemargin}{0pt}
\setlength{\evensidemargin}{0pt}
\setlength{\textheight}{46\baselineskip}
\setlength{\textwidth}{47zw}

\title{数学$IA$ 基本問題}
\date{}
\author{}

\begin{document}
\maketitle

\begin{itembox}{始めに}
\begin{flushleft}
\begin{itemize}
\item この問題はあくまで「入試から見た基本問題」で、実際的には章末問題レベルを想定しています。単に答えを求めるだけではなく、いかに相手に伝わりやすいスマートな解答を書けるかにこだわって取り組んでください。(つまり、単に値を求めるだけではダメです。)\\
\item 目標解答時間は設定しませんが、あまりかけすぎるのはよくないでしょう。また、入試基礎の問題なので、教科書等は見ず、自力で取り組みましょう。\\
\item $1$度解くだけで終わるのは勿体ないです。少なくとも間違った問題については何度でも取り組み、完璧に解答できるようにしておきましょう。それだけの値のある問題たちばかりです。
\item この問題の他にもっと難度の高い問題も用意します。それらも含め、こちらの準備が出来次第問題は随時配布します。\\
\end{itemize}
\end{flushleft}
\end{itembox}

さあ、北大の理系数学を$5$題完答できるようになるつもりで問題に取り組みましょう!
そして、数学を絶対的な得意科目にしましょう!


\section{数と式}

\begin{enumerate}
\item 次の式を展開せよ。
\begin{enumerate}[(1)]
\item $(x+1)^4-(x-1)^4$
\item $(x+1)(x-2)(x^2-x+1)(x^2+2x+4)$
\end{enumerate}\

\item 次の式を因数分解せよ。
\begin{enumerate}[(1)]
\item $a^2(b-c)+b^2(c-a)+c^2(a-b)$
\item $a^2(b+c)+b^2(c+a)+c^2(a+b)$
\item $a^2-ab-2b^2-a-7b-6$
\end{enumerate}\

\item $x=\sqrt{2}+3$のとき、$x^3-9x^2+27x-27$の値を求めよ。\newline

\item $x=2a+1$のとき、$\sqrt{x^2-8a}+\sqrt{a^2+x}$をaで表せ。\newline

\item $\displaystyle\frac{1}{1-\sqrt{2}+\sqrt{3}}$の分母を有理化せよ。\newline

\item $a=\displaystyle\frac{\sqrt{6}+\sqrt{2}}{2}$のとき、$a^2+\displaystyle\frac{1}{a^2}$、$a^3+\displaystyle\frac{1}{a^3}$、$a^4-\displaystyle\frac{1}{a^4}$の値をそれぞれ求めよ。\newline

\item $\sqrt{5}$の整数部分を$a$、小数部分を$b$とするとき、$\displaystyle\frac{a}{b}$の小数部分を求めよ。\newline
\end{enumerate}

\section{方程式と不等式}

\begin{enumerate}
\item $a$を定数とする。$x$の不等式 \[ax>x+a^2-1 \]について、各問いに答えよ。
\begin{enumerate}[(1)]
\item $a=-3$、$a=3$のときの不等式の解をそれぞれ求めよ。
\item 不等式の解を求めよ。 
\end{enumerate}\

\item $x$の$2$つの不等式 
\begin{enumerate}
\item $|x-5|<3$
\item $|x-k|<2$ 
\end{enumerate}
\begin{flushleft}
について、次の各問いに答えよ。
\end{flushleft}
\begin{enumerate}[(1)]
\item (a)、(b)をともに満たす実数$x$が存在しないように、定数$k$の値の範囲を求めよ。
\item (b)の解が(a)に含まれるように、定数$k$の取り得る値の範囲を求めよ。
\end{enumerate}\

\item パーティの食事代を支払うのに、$1$人$2000$円ずつ支払うと$3600$円足りなかった。そこで、$1$人$2500$円ずつ支払うことにしたが、まだ$1000$円以上足りないので、ある$1$人が$2500$円ではなく$4000$円支払ったところ、おつりが出た。パーティの食事代を求めよ。\newline

\item 次の不等式を解け。
\begin{enumerate}[(1)]
\item $2x-4<2<|x|$ 
\item $x+1<|2x-2|<3x+1$ 
\end{enumerate}\

\item 次の方程式を解け。
\begin{enumerate}[(1)]
\item $3x^4-2x^2-8=0$ 
\item $(x^2+2x)^2-2(x^2+2x)-3=0$ 
\end{enumerate}\

\item $2$次方程式 
\begin{enumerate}
\item $x^2+4x+n+1=0$
\end{enumerate}
\begin{flushleft}
について、次の各問いに答えよ。
\end{flushleft}
\begin{enumerate}[(1)]
\item (a)が実数解をもつとき、$n$のとり得る値の範囲を求めよ。
\item $n$を自然数とする。(a)の実数解が全て整数となるように$n$の値を定め、そのときの解を求めよ。
\end{enumerate}\

\item $x$の$2$次方程式 
\begin{enumerate}
\item $x^2-x-6=0$
\item $x^2+a^2x+a-7=0$
\end{enumerate}
\begin{flushleft}
について、各問いに答えよ。
\end{flushleft}
\begin{enumerate}[(1)]
\item 方程式(a)を解け。
\item 方程式(a)、(b)がただ$1$つの共通解をもつような定数$a$の値を求めよ。
\end{enumerate}\

\item {$\bigtriangleup ABC$は\[ \angle CAB=36^\circ 、AB=AC=1 \] の二等辺三角形である。
\begin{enumerate}[(1)]
\item $\angle ABC$の2等分線とACとの交点をDとするとき、$\bigtriangleup ABC\sim\bigtriangleup BCD$を証明せよ。\newline (注:「$\sim$」は相似の記号)
\item BCの長さを求めよ。
\end{enumerate}
}
\end{enumerate}

\section{$2$次関数}

\begin{enumerate}
\item 放物線$y=x^2-4x+m$を原点に関して対称移動し、次に$x$軸方向に$2p$、$y$軸方向に$p$だけ平行移動した放物線は、点$(4,0)$で$x$軸と接する。$m$および$p$の値を求めよ。\newline

\item グラフが次の条件をみたすとような$2$次関数を求めよ。
\begin{enumerate}
\item 2点$(2,-1)$、$(5,-4)$を通り、$x$軸に接する。
\item 点$(4,-10)$を通り、軸の方程式が$x=1$で、グラフと$x$軸との$2$つの交点の距離が$4$である。
\item $x$軸方向に$2$、$y$軸方向に$3$だけ平行移動すると、$3$点$(0,11)$、$(2,3)$、$(5,6)$を通る。
\end{enumerate}\

\item $a\neq 0$とする。$2$次関数$y=ax^2-ax+b$の$0\leq x\leq 3$における最大値が$4$、最小値が$-2$であるとき、$a$、$b$の値の組をすべて求めよ。\newline

\item 放物線$y=16-4x^2$と$x$軸とで囲まれた部分に内接する長方形を作る。ただし、長方形の$1$辺は$x$軸上にあるものとする。この長方形の周の長さの最大値を求めよ。\newline

\item $x$、$y$が$x^2+2y^2=2$をみたすとき、$4x+4y^2$の最大値と最小値、およびそのときの$x$、$y$の値を求めよ。\newline

\item 連立不等式
\begin{numcases}
{}
x^2-4>0 \\
x^2-(a+1)x+a \leq 0
\end{numcases}
\begin{flushleft}
について、次の各問いに答えよ。
\end{flushleft}
\begin{enumerate}
\item 不等式(2)を解け。
\item 連立不等式($1$),($2$)をみたす実数$x$が存在するような$a$の値の範囲を求めよ。
\item 連立不等式($1$),($2$)をみたすような整数$x$がただ$1$つ存在するような、$a$の値の範囲を求めよ。
\end{enumerate}\

\item 次の各問いに答えよ。
\begin{enumerate}
\item $2$次不等式$2x^2-ax+2a-6>0$の解が、すべての実数となるとき、定数$a$の値の範囲を求めよ。
\item $2$次関数$y=x^2+(2m+1)x+2m-1$のグラフは、定数$m$の値に関係なくつねに$x$軸と異なる$2$点で交わることを示せ。
\end{enumerate}\

\item $x\geq 0 $において、不等式$x^2-2ax+a+12$が成り立つような$a$の値の範囲を求めよ。\newline

\item $x$の$2$次方程式$x^2-2x+a^2-5=0$が次の解をもつようなaの値の範囲を求めよ。
\begin{enumerate}
\item $-1<x<2$の範囲に異なる$2$つの解をもつ。
\item 異符号の解をもつ。
\end{enumerate}

\end{enumerate}







\end{document}