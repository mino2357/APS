\documentclass[10pt,a4paper,fleqn]{jsarticle}

\usepackage{amsmath,amssymb}
\usepackage{bm}
\usepackage{graphicx}
\usepackage{ascmac}
\usepackage{enumerate}
%
\setlength{\topmargin}{-0.3in}
\setlength{\oddsidemargin}{0pt}
\setlength{\evensidemargin}{0pt}
\setlength{\textheight}{46\baselineskip}
\setlength{\textwidth}{47zw}

\title{数学$IA$ 発展問題}
\date{}
\author{}

\begin{document}
\maketitle

\begin{itembox}{始めに}
\begin{flushleft}
\begin{itemize}
\item この問題は「入試を想定した問題」で、実際に入試に出た問題から出題しています。問題を解くときは自分も試験を受けるつもりで、いかに相手に伝わりやすいスマートな解答を書けるかにこだわって取り組んでください。\\
\item 目標解答時間は設定しません。難易度は決して低くはないので、場合によっては参考書等でヒントを探すことも良いでしょう。ただし、少なくとも1つの問題に15分以上自力で取り組むようにしてください。わからないとすぐに解答を見ようとするのはあまりに愚かです。\\
\item $1$度解くだけで終わるのは勿体ないです。少なくとも間違った問題については何度でも取り組み、完璧に解答できるようにしておきましょう。それだけの値のある問題たちばかりです。
\end{itemize}
\end{flushleft}
\end{itembox}

さあ、北大の理系数学を$5$題完答できるようになるつもりで問題に取り組みましょう!
そして、数学を絶対的な得意科目にしましょう!


\section{数と式}
\begin{enumerate}
\item 次の式を因数分解せよ。 (1)福岡教育大、(2)東海大、(3)名古屋経大
\begin{enumerate}[(1)]
\item $(ab+1)(a+1)(b+1)+ab$
\item $(b-a)^3-(b-c)^3-(c-a)^3$
\item $x^4-3x^2y^2+y^4$
\end{enumerate}\

\item $k$を定数とする$2$次式$x^2+3xy+y^2-3x-5y+k$が$x$、$y$の$1$次式の積に因数分解できるとき、$k$の値を求めよ。また、そのときの因数分解の結果を求めよ。 (東京薬大・改)\newline

\item $\displaystyle\sqrt{14+6\sqrt{5}}$の整数部分を$a$、小数部分を$b$とする。このとき \newline ~~~~~~$a$ ,  $b^2+\displaystyle\frac{1}{b^2}$ ,  $b^3+\displaystyle\frac{1}{b^3}$ \\* の値を求めよ。 (青山学院大)\newline

\item 方程式$18x-5y=1$を満たす自然数の解の組($x,y$)を求めよ。 (昭和女子大・改)\newline

\item どのような整数$n$に対しても$n^2+n+1$は$5$で割り切れないことを示せ。 (学習院大)\newline

\item 有理数$p、q、r$について、$p+q\sqrt{2}+r\sqrt{3}=0$ならば、$p=q=r=0$であることを示せ。ただし$\sqrt{2}、\sqrt{3}、\sqrt{6}$が無理数であることは使ってよい。 (京都大)\newline

\end{enumerate}

























\end{document}